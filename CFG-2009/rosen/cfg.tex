\documentclass[a4paper,12pt]{article}
\usepackage[A4]{vmargin}
%% ltrb
\setmargnohfrb{2.5cm}{2.5cm}{2.5cm}{2.5cm}
\pagestyle{empty}
\usepackage[british]{babel}
\usepackage[utf8x]{inputenc}
\usepackage[T1]{fontenc}
\usepackage{times}
\usepackage{sectsty,layout}
\allsectionsfont{\mdseries\normalsize\itshape}
\usepackage{linguex}
\usepackage{gloss}

% This is how linguex is used:
% \setcounter{ExNo}{\value{enums}}
% \ex.
% \ag. Přišla by.\\
%       come{\sc -pple} {\sc cond.aux}\\
%       `She would come' \label{prislaby}
% \bg. Mohla  by přijít.\\
%      {can{\sc -pple}}  {{\sc cond.aux}} come{\sc -inf}\\
%      `She could come' \label{mohlabyprijit}
% \setcounter{enums}{\value{ExNo}}

\newcommand{\opis}[1]{ \hspace*{\fill}\mbox{\footnotesize #1}}

\usepackage{natbib}
\setlength{\bibsep}{2.1pt}

\makeatletter
\newcommand\textsubscript[1]{\@textsubscript{\selectfont#1}}
\def\@textsubscript#1{{\m@th\ensuremath{_{\mbox{\fontsize\sf@size\z@#1}}}}}
\newcommand\textbothscript[2]{%
  \@textbothscript{\selectfont#1}{\selectfont#2}}
\def\@textbothscript#1#2{%
  {\m@th\ensuremath{%
    ^{\mbox{\fontsize\sf@size\z@#1}}%
    _{\mbox{\fontsize\sf@size\z@#2}}}}}
\def\@super{^}\def\@sub{_}

\catcode`^\active\catcode`_\active
\def\@super@sub#1_#2{\textbothscript{#1}{#2}}
\def\@sub@super#1^#2{\textbothscript{#2}{#1}}
\def\@@super#1{\@ifnextchar_{\@super@sub{#1}}{\textsuperscript{#1}}}
\def\@@sub#1{\@ifnextchar^{\@sub@super{#1}}{\textsubscript{#1}}}
\def^{\let\@next\relax\ifmmode\@super\else\let\@next\@@super\fi\@next}
\def_{\let\@next\relax\ifmmode\@sub\else\let\@next\@@sub\fi\@next}
\makeatother

\newcommand{\cat}[1]{_{\textbf{\textsc{#1}}\ }}
\newcommand{\gram}[1]{{_{\textbf{\textsc{#1}}}\ }}


\begin{document}

\setlength{\Exlabelsep}{0em}
\setlength{\SubExleftmargin}{1.2em}


% \layout{}

% \maketitle

\noindent
Jarmila Panevová and Alexandr Rosen (Charles University, Prague)

\bigskip{}

\noindent
{\large Source of case agreement for secondary predicates in Czech non-finite clauses} 

% \bigskip{}


\section{The problem}


In simple finite clauses, forms of adjectival declension playing the role of
predicative elements agree with their subject\footnote{Note that
  the subject of a secondary predicate is not necessarily identical
  with the subject of the primary predicate, as in \ref{fin:acc}.} as
expected: in gender, number, and, crucially, in case
\ref{fin:nom}--\ref{fin:gen}.\footnote{This is true also about quantified
nominals in genitive plural with a distinct agreement pattern
\ref{fin:gen}.} The only alternative of the fully
agreeing adjectival form is a form showing the non-agreeing
instrumental case, supposed to denote a transient quality, but quite
often a mere stylistic marker. !!příklad?


\ex. \ag. \emph{Venkovští} \emph{kluci} \emph{běhají} \emph{bosi}.  \\
village boys\cat{nom} run barefoot\cat{nom} \\
`Village boys run barefoot.' \label{fin:nom}
     \bg. \emph{Přivedl} \emph{poníka} \emph{do} \emph{stáje} \emph{uříceného}. \\
           pony\cat{acc} brought into stable hot\cat{acc} \\
           `He brought the pony into the stable covered in foam.'
            \label{fin:acc}
     \bg. \emph{Hodně} \emph{turistů} \emph{se} \emph{vrátilo} \emph{z} \emph{dovolené} \emph{nespokojených}. \\
many tourists\cat{gen,pl} \textsc{refl} returned from holiday dissatisfied\cat{gen,pl} \\
`Many tourists came back from holiday dissatisfied.'\label{fin:gen}



\section{Known data}

Predicative complement of an infinitive (or, more generally, secondary
predicate in any non-finite clause) is usually assumed to agree in all
relevant categories, including case, with its (the complement's)
subject, itself in the position of subject, object, or some other
dependent of the infinitive.\footnote{Again, except for the option of
  the non-agreeing instrumental case. Because the availability of this
  option obeys the same rules both in finite and in non-finite
  clauses, it will be disregarded in the following.} The only
difference between a finite and non-finite form is the unavailability
of overt infinitival subject. Precisely these structures, with
predicative complements referring to covert infinitival subjects, are
in our focus.

It follows from the assumption above that covert infinitival
subject bears morphological case, identical with the case of
predicative complement. The question of predicative complement's
agreement can then be asked in a different way: as a question about
covert subject's case assignment. Here, the controlling or raised
constituent referring to the same entity as the infinitival subject
may play a crucial role. For brevity, this constituent will be called 
\emph{antecedent}.\footnote{We are aware that the term
  \emph{antecedent} is justified only for control structures, where
  PRO indeed has an antecedent, and hardly at all for raising/ECM
  structures, where there is no co-referring
  pronominal. \cite{huds:03} coins the term \emph{anchor} to cover both.}

The covert infinitival subject can be assigned case in two ways: 

\begin{enumerate}
\item Independently of the antecedent's case, according to
  language-specific principles, resulting in \emph{\textbf{C}ase
    \textbf{I}ndependence}. These principles sometimes need to be informed
  by lexical specifications of the embedded verb.
\item In accordance with the case of the antecedent, resulting in
  \emph{\textbf{C}ase \textbf{T}ransmission}.
\end{enumerate} 



\subsection{Ancient Greek}


Unfortunately, this neat correlation breaks down when faced with
data. In Ancient Greek, where the independent case of the covert
subject is accusative, only raising-to-subject (SR) construction shows
the expected pattern, namely CT \ref{kyros}. Contrary to theoretical
wisdom, subject control constructions have CI rather than CT
\ref{dareios}, and in ECM and object control (OC) both CI and CT is
acceptable \ref{prothymon}, \ref{poiesanta}.\footnote{Original examples quoted after \cite{huds:03}.}

\ex. \ag. \emph{Dáreios} \emph{búletai} \emph{polemikos} / \emph{*polemikon} \emph{einai}.\\
          Darius\gram{{nom}} wants warlike\gram{{nom}} {\ } warlike\gram{acc} {to be} \\
          `Darius wants to be warlike.' \opis{\citealt[p.~102]{laca:78}; SC, CT: nom--nom} \label{dareios}
     \bg. \emph{dokei} {\emph{{{ho}}}} {\emph{{{Kýros}}}} {\emph{einai}} {\emph{{sofos}}} / \emph{*sofon.} \\
          seems {\textsc{def}\gram{nom}} {Cyrus\gram{{nom}}} {to be} {wise\gram{{nom}}} {\ } wise\gram{acc} \\
          `Cyrus seems to be wise.' \opis{\citealt[p.~127]{Quicoli:82}; SR, CT: nom--nom} \label{kyros}
     \bg. \emph{symbúleuó} {\emph{{{soi}}}} {\emph{{prothymói}}} / {\emph{{prothymon}}} {\emph{einai}}.\\
          {I advise} {you\gram{{dat}}} {zealous\gram{{dat}}} {\ } {zealous\gram{{acc}}} {to be}\\
          `I advise you to be zealous.' \opis{\citealt[p.~106]{laca:78}; OC, CT/CI: dat--dat/acc} \label{prothymon}
     \bg. {\emph{synoida}} {\emph{{{soi}}}} \emph{eu} {\emph{{poiésanti}}} / \emph{{poiésanta}}. \\
          {I know} {you\gram{{dat}}} {well} {having-done\gram{{dat}}} {\ } {having-done\gram{{acc}}} \\
          `I know you have done well.' \opis{\emph{ibid.} p.~107; ECM, CT/CI: dat--dat/acc} \label{poiesanta}


Ancient Greek thus presents the following picture. The unexpected patterns are in bold type:\footnote{Throughout this paper, the right pointing arrows $\longrightarrow$ are used to denote implication with the following proviso: the consequent is listed exhaustively, i.e., ``SC $\longrightarrow$ \textbf{CT}'' means that in SC only CT is possible.}

  \begin{itemize}
  \item SC $\longrightarrow$ \textbf{CT}
  \item SR $\longrightarrow$ CT
  \item OC $\longrightarrow$ CI/\textbf{CT}
  \item ECM $\longrightarrow$ \textbf{CI}/CT
  \end{itemize}

\subsection{Russian}

  In Russian, the independent case is dative. As in Ancient Greek, we
  find CT in SC \ref{jap} and, optionally, in OC
  \ref{samomu}. Interestingly, not all cases seem to behave in the
  same way: OC with genitive antecedent does not allow for CT
  \ref{samix}.\footnote{See, e.g., \cite{Comrie:74} and
    \cite{Landau:07}. Russian has no ECM structures.} !!SR example?

  
\ex. \ag. {\emph{{{Ona}}}} {\emph{sobiralas'}} {\emph{putešestvovat'}} {\emph{{odna}}} / {*odnoj} {v} {Japonii.} \\
          {she\gram{{nom}}} {planned} {to travel} {alone\gram{{nom}}} {\ } {alone\gram{dat}} {in} {Japan} \\
          `She planned to travel alone in Japan.' \opis{\citealt[p.~888]{Landau:07}; SC, CT: nom--nom}  \label{jap}
     \bg. \emph{Ona} {\emph{ugovorila}} {\emph{{{ego}}}} {\emph{pogovorit'}} {\emph{{samogo}}} / {\emph{{samomu}}} \emph{s} \emph{ejo} \emph{roditeljami.} \\
          {she\gram{{nom}}} {convinced} {him} {to talk} {himself\gram{{acc}}} / {himself\gram{{dat}}} {with} {her} {parents} \\
          `She convinced him to talk himself to her parents.' \opis{\emph{ibid.}; OC, CT/CI: acc--acc/dat} \label{samomu}
     \bg. \emph{Dlja} {\emph{{{nas}}}} {\emph{utomitel'no}} {\emph{delat'}} \emph{eto} {\emph{{samim}}} / \emph{*samix}. \\
          {for} {us\gram{{gen}}} {tiresome} {to do} {that} {ourselves\gram{{dat}}} {\ } {ourselves\gram{gen}} \\
          `It is tiresome for us to do this ourselves.' \opis{\emph{ibid.} p.~893; OC, CI: gen--dat} \label{samix}

The overall situation in Russian is the following:

\begin{itemize}
  \item SC $\longrightarrow$ \textbf{CT}
  \item SR $\longrightarrow$ CT
  \item OC acc $\longrightarrow$ CI/\textbf{CT}
  \item OC gen, ?dat $\longrightarrow$ CI
  \item ECM $\longrightarrow$ n/a
  \end{itemize}

  However, the picture in Russian is more complex, providing further
  evidence against the simplistic view reducing the factors to the
  contrast between control and raising/ECM. Unlike in \ref{samix}, CT
  is possible with genitive of negation and OC, as in
  \ref{tarelki}. CI is possible in passive even with nominative
  antecedent \ref{bystro}. !!Expand: Shows that SC/OC/SR is a
  misnomer... Exceptions to the SC $\longrightarrow$ CT
  generalizations include clauses with an intervening Addressee (Goal)
  \ref{zavtra}, where both CT and CI is possible, and an intervening
  wh- item \emph{kak} \ref{nacalnik}, where only CI is allowed..

\exg. \emph{On} \emph{ne} \emph{naučil} \emph{ni} \emph{{{odnogo}}} \emph{{{ditja}}} \emph{est'} \emph{{samogo}} / \emph{{samomu}} \emph{iz} \emph{tarelki}.\\
he not taught no one child\cat{gen} eat itself\cat{gen} {\ } itself\cat{dat} from plate \\
`He didn't teach any child to feed herself from the plate.' \\
\opis{\citealt[p.~905]{Landau:07}; OC, CT/CI: gen--gen/dat} \label{tarelki}

% \item V pasivu s nominativnim antecedentem lze CT i CI

\exg. \emph{{{On}}} \emph{byl} \emph{poprošen} \emph{imi} \emph{sdelat'} \emph{eto} \emph{\emph{{odin}}} / \emph{{odnomu}} \emph{bystro}.\\
he\cat{nom} was requested {by them} {to do} it alone\cat{nom} {\ } alone\cat{dat} quickly \\
`He was requested by them to do it alone quickly.' \opis{\emph{ibid.}; SC, CT/CI: nom--nom/dat} \label{bystro}

% \item SC s adresátem (GOAL) umožňuje CT i CI.

\exg. \emph{{{Ivan}}} \emph{pokljalsja} \emph{druzjam} \emph{sdelat'} \emph{eto} \emph{{sam}} / \emph{{samomu}} \emph{zavtra}.\\
Ivan\cat{nom} vowed friends {to do} it himself\cat{nom} {\ } himself\cat{dat} tomorrow \\
`Ivan vowed to his friends to do it alone tomorrow.'\opis{\emph{ibid.}, p.~890; SC, CT: nom--nom/dat}
 \label{zavtra}


% \item SC s infinitivem uvozeným \textit{kak} připouští pouze CI.

\ex. \label{nacalnik}
     \ag. \emph{{{On}}} \emph{zabyl} \emph{pogovorit'} \emph{{sam}} / \emph{*samomu} \emph{s} \emph{načal'nikom}.\\
          he\cat{nom} forgot {to talk} himself\cat{nom} {\ } himself\cat{dat} with boss\\
          `He forgot to talk himself to the boss.'\opis{\emph{ibid.}, p.~893; SC, CT: nom--nom}
     \bg. \emph{{{On}}} \emph{zabyl} \emph{kak} \emph{pogovorit'} \emph{*sam} / \emph{{samomu}} \emph{s} \emph{načal'nikom}.\\
         he\cat{nom} forgot how {to talk} himself\cat{nom} {\ } himself\cat{dat} with boss\\
          `He forgot how to talk himself to the boss.'\opis{\emph{ibid.}; SC, CI: nom--dat}



% \bibliographystyle{natbib}
\bibliographystyle{natbib}
\bibliography{/home/rosen/BIBLIOGRAPHY/unibib.bib}



\end{document}

