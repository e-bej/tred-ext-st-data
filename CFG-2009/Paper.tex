%!TEX TS-program = pdfcslatex
%!TEX encoding = UTF-8 Unicode
\documentclass[12pt]{article}
\usepackage{geometry}                % See geometry.pdf to learn the layout options. There are lots.
\geometry{a4paper}                   % ... or a4paper or a5paper or ... 
%\geometry{landscape}                % Activate for for rotated page geometry
\usepackage[parfill]{parskip}    % Activate to begin paragraphs with an empty line rather than an indent
\usepackage{amsfonts}
\frenchspacing
%\sloppy
\usepackage{czech}
\usepackage[utf8x]{inputenc} % pre-august 2004 use is \usepackage[utf8]{inputenc}
\usepackage[T1]{fontenc}
\usepackage{lmodern}
\usepackage{euler} % math font
\usepackage{paralist} % for better itemize and enumerate
% asparaenum, inparaenum, \begin{compactenum}[{Example} a)] etc.
% Also, items can be referenced via \label{} and \ref{}
%
\usepackage{array}
%\usepackage{longtable}
\usepackage{float}
\usepackage{natbib}
%\usepackage[
%	bookmarks,
%	bookmarksopen,
%	pdftitle=Representace\ víceslovných\ výrazů\ v\ Pražském\ závislostním\ korpusu,
%	pdfdisplaydoctitle,
%	pdfauthor=Eduard\ Bejcek\ and\ Pavel\ Stranak,
%	pdfcreator=PdfLaTeX\ with\ Hyperref\ package,
%	colorlinks,
%	unicode,
%]{hyperref}
\pagestyle{empty} % Remove a header, line number etc.


\title{Representace víceslovných výrazů\\ v Pražském závislostním korpusu}
%\author{Pavel Straňák}
\date{}                                           % Activate to display a given date or no date
%%%%%%%%%%%%%%%%%%%%%%%%%%%%%%%%%%%%%%%%%%%%%%%
%%%%%%%%%%%%%%%%%%%%%%%%%%%%%%%%%%%%%%%%%%%%%%%
\begin{document}


\maketitle
\thispagestyle{empty} % Remove a header, line number etc. on this page.

\section*{Abstrakt}
Víceslovné výrazy (idiomy, frazémy, víceslovná vlastní jména ad., VV) stojí často na okraji zájmu při utváření syntaktických teorií. V důsledku toho jsou pak v~syntaktických stromech representovány sporným až arbitrárním způsobem.

V tomto článku nejprve popíšeme stav anotace víceslovných výrazů v tektogramatickém popisu v Pražském závislostním korpusu verse 2.0 (Prague Dependency Treebank 2.0, PDT), ~\citep{hajic:2005}). Rozebereme každý funktor identifikující některý druh víceslovných výrazů i případy, ve kterých víceslovné výrazy nejsou explicitně identifikovány~\citep{mikulova:2006}. Zmíníme se také o shodách a rozdílech teoretického funkčního generativního popisu~\citep{sgall-etal:1986} a pozdější praktické implementace v PDT.
\begin{table}[htdp]
\begin{center}
\begin{tabular}{|l|l|}
\hline
t-node & 528504 \\
CPHR & 2221\\
DPHR & 978 \\
FPHR & 3620 \\
t-lemma = \#Forn & 1310 \\
t-lemma = \#Idph & 702 \\
ID & 5250 \\
ID (parent = Idph) & 654 \\
\hline
\end{tabular}
\caption{Frekvence v PDT 2.0}
\end{center}
\label{default}
\end{table}%

DPHR - spojené: ``V případě, že má frazém více závislých částí, vystupují v t-lematu uzlu s funktorem DPHR všechny tyto formy spojené podtržítkem v pořadí, v jakém se vyskytovaly v povrchové podobě věty'' \citep{mikulova:2006}. Naproti tomu analyzované názvy firem v ID, dokonce příklady stromů ve Fonr, to je ale proti instrukci v manuálu. Nekonzistence cizích jmen firem jako FPHR  s víceslovným cizím jménem města (příklad v Libanonu) jako ID.

Najít některé zajímavé jevy je kvůli podspecifikovanosti funktoru ID nemožné. Např název knihy ``Nezaměstnanost jako sociální problém''. <obr.>

Dále popíšeme projekt, který si klade za cíl zlepšit identifikaci VV v PDT. Probereme anotační pravidla i metodu anotace, jež spojuje ruční práci s pokročilými automatickými metodami vyhledávání VV bez ohledu na jejich tvar. Popíšeme použitou representaci VV ve slovníku i v anotovaných datech a vztah nové anotace k současným rovinám PDT.

Stručně probereme problematiku shody anotátorů na identifikaci a potažmo na bližším určení VV. S tím souvisí %velmi zajímavá 
problematika měření mezianotátorské shody (viz např. ~\citet{artstein:2007}). Vysvětlíme, proč jsme pro tento úkol vyvinuli vlastní metriku měření a naznačíme její vlastnosti.

V návaznosti na výše uvedený projekt probereme změny v representaci VV navrhované pro příští versi PDT. Na závěr porovnáme stávající i námi navrhovaný upravený přístup PDT k víceslovným výrazům s representací použitou v jiných syntakticky anotovaných korpusech.

\bibliographystyle{plainnat}
\newcommand{\bibfont}{\small}
\bibliography{Bibliografie}
\end{document}   
