% !TEX root = ../disertace.tex
%!TEX encoding = UTF-8 Unicode

\chapter{Conclusion}
\label{sec:conclusion}
\todo

\section{z LRE}

We have annotated multi-word lexemes and named entities on a part of PDT 2.0. We use tectogrammatical tree structures of MWEs for automatic pre-annotation. In Section~\ref{sec:annot:pre} we show that the richer the tectogrammatical annotation the better the possibilities for automatic pre-annotation that minimizes human errors. In the analysis of inter-annotator agreement we show that a weighted measure that accounts for partial agreement as well as estimation of maximal agreement is needed. 

The resulting $\kappa_w^U = 0.644$ is statistically significant and should gradually improve as we clean up the annotation lexicon, more entries are pre-annotated automatically, and further types of pre-annotation are employed.

The metodology of MWE annotation we developed enables precise pre-annotation by automatically extracted tectogramatical tree structures. We have shown that this pre-annotation does improve annotation speed and should improve also agreement. Initially only manual annotation will become more automatic and our pre-annotation procedures will mark more occurances. This leaves the annotator only to decide weather this meaning is correct or not.

---------

\section{rozpracovat}
\xx{Co jsme udelali:} hypoteza o tekto stromech MWE, anotacni nastroj, datovy format pro data, vyuziti hypotezy v predanotaci, \semlex, jeho datova representace, t-stromecky v semlexu, extension -- zobrazeni anotaci a vyhledavani v nich; prvni data, o kterych vime, kde muze i uzivatel hledat shodu a neshodu, apod. (umozneno skvelymi nastroji jinych: PML, tred, pml-tq). Hypoteza potvrzena? Mimochodem anotaci nalezeny chyby v PDT, nektere systematickeho razeni (chybejici uzly v koordinacich?), a take se ukazalo, ze nas blokovaly nedokoncenosti t-lemmat (deminutiva, prechylovani).

\xx{Zhodnotit vysledky toho, ze nekdy se vyplati anotovat rucne}, a nekdy ne: pripad jmen osob. Zjevne jsme udelali chybu, ze jsme to neanalyzovali takto (viz data od PaSi - EB) drive a nepredanotovali zbytek automaticky podle PDT. Ovsem, nasli by pak anotatori to, co nasli, kdyz by si zvykli na velmi vysokou uspesnost predanotace?

\xx{Co jsme nacali:} logy a jejich analyzy -- pokud vime, nikdo takto podrobne udaje nema, pouziti k urceni ceny anotaci, analyza toho, co se pri anotaci vlastne doopravdy deje: jake promenne hraji roli, jake jsou mezi nimi vztahy? 

\xx{Pravdepodobnostni pristup k idiomaticnosti} -- mira, nikoliv kategorialni velicina s hodnotami nic, frazem, idiom, ...
