% !TEX root = ../disertace.tex
%!TEX encoding = UTF-8 Unicode

\chapter{Conclusion}
\label{sec:conclusion}
We have annotated multi-word lexemes and named entities on a part of PDT 2.0. We use tectogrammatical tree structures of MWEs for automatic pre-annotation. In Section~\ref{sec:annot:pre} we show that the richer the tectogrammatical annotation the better the possibilities for automatic pre-annotation that minimizes human errors. In the analysis of inter-annotator agreement we show that a weighted measure that accounts for partial agreement as well as estimation of maximal agreement is needed. 

The resulting $\kappa_w^U = 0.644$ is statistically significant and should gradually improve as we clean up the annotation lexicon, more entries are pre-annotated automatically, and further types of pre-annotation are employed.

The metodology of MWE annotation we developed enables precise pre-annotation by automatically extracted tectogramatical tree structures. We have shown that this pre-annotation does improve annotation speed and should improve also agreement. Initially only manual annotation will become more automatic and our pre-annotation procedures will mark more occurances. This leaves the annotator only to decide weather this meaning is correct or not.