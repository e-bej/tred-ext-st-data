% !TEX root = ../disertace.tex
%!TEX encoding = UTF-8 Unicode

\chapter{Idioms}
Even ``non-compositional'' idioms are actually (originaly) metaphorical or methonymical.  Even though sometimes it is hard to see that. At other times a speaker may forget that rather straightforward metaphoric aspect:

\begin{quote}
Barack Obama accused his Republican rivals of stirring a controversy over a comment he made about putting “lipstick on a pig.” \emph{(NY~Times, 11.~September 2008)}
\end{quote}

\section{Sinclair}
Patrick Hanks gives a very succinct introduction to John Sinclair's \citep{sinclair:wiki} view of a lexical unit and meaning distinctions. It is closely related to Hanks' own work, as we cen see in \citet{hanks:norms-and-exploitations}

\section{Žabokrtský}
Zdeněk Žabokrtský defines a \emph{lexical unit} in his doctoral thesis \citep{zabokrtsky:2005a}. His definition is based on the works of \citet{cermak:91} and \citet{filipec:1994}. His view however difers slightly from the view of the above mentioned authors. Žabokrtský defines a \emph{lexical unit} as

\section{ToAdd}
Pecina, Cruse, Filipec, Čermák (rozdelit, nebo spojit se ZŽ?), Hanks. Zminit clanek SC, MH a Lenky?? o Hanksove CPA jako realizaci Sinclairova pristupu k lexikalnimu vyznamu.

Valencni slovniky a \emph{anotace valence}.