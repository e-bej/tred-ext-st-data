% !TEX root = ../disertace.tex
%!TEX encoding = UTF-8 Unicode

\chapter{Idioms}
Even ``non-compositional'' idioms are actually (originally) metaphorical or metonymical.  Even though sometimes it is hard to see that. At other times a speaker may forget that rather straightforward metaphoric aspect:

\begin{quote}
Barack Obama accused his Republican rivals of stirring a controversy over a comment he made about putting “lipstick on a pig.” \emph{(NY~Times, 11.~September 2008)}
\end{quote}

\section{Sinclair}
Patrick Hanks gives a very succinct introduction to John Sinclair's \citep{sinclair:wiki} view of a lexical unit and meaning distinctions. It is closely related to Hanks' own work, as we can see in \citet{hanks:norms}.

\section{Žabokrtský}
Zdeněk Žabokrtský defines a \emph{lexeme} in his doctoral thesis \citep{zabokrtsky:2005a}, as well as several other basic concepts that he requires for precise description of his work on valency lexicon of Czech verbs. His definition is based on the concept of lexeme as defined in works of \citet{cermak:91} and \citet{filipec:1994} and a concept of lexical unit as defined by \citet{cruse:1986}\footnote{Žabokrský cites \citealp{verspoor:1997}'s description of Cruse's lexeme, but that makes no difference {\xxx(check!)}}. Žabokrtský's view seems to differ from the view of the above mentioned authors, but that difference deserves some analysis.

Žabokrtský defines a lexeme as ``an ordered pair which couples the set of lexical forms and the set of lexical units''. This way a definition of lexeme is based on a definition of the concepts ``lexical form'' and ``lexical unit''. It is also implied that these are somehow complementary. 

A lexeme is however a relatively common concept in lexicography. It has been defined at least by Lyons, by Cruse, Čermák, Filipec, and probably by many other authors. What is common to all of these definitions is that they construe lexeme as a unit consisting of some lexical form and meaning. 
\emph{It is in fact interesting to try to find what distinguishes a concept of lexeme from a concept of concept itself! It seems that a concept is also a meaning (essentia) with a form.} See \citet{materna:1998,stranak:2001} for detailed discussion of definition of concepts as well as thorough explanation (in the latter) of the reasons why we do not delve into the problem of concept itself too deeply: it is out of the methodological reach of modern science as we understand it: the problem of what is a concept cannot be solved by any conceptual analysis.

A \emph{lexical unit} in turn is used as defined by Verspoor referring to \citet{cruse:1986}.\footnote%
{\xxx Najit Crusovu definici lexemu a lex. unit a analyzovat. Ma tam MWE, nebo NE??} 

%% ME %%
\section{Straňák}
Concept vs. word/lexeme/lexia\\
ergo\\
Language vs. conceptual system\\
Is language in fact the conceptual system? If not, then why? What is the difference? 

\subsection{How about translations}
Are they a part of a lexeme? Why not?

To Žabokrtský, but also all(?) the lexeme definitions: A ZZ lexeme is an ordered (why??) pair composed of a set of forms and a set lexical units (meanings in his understanding?). There is no limit on forms that can be a part of a lexeme, so there seems to be no reason to exclude the forms of a different language, as long as they fit the ``lexical units''. LU (or a lexeme of Filipec and Cruse(?)) is again a pair: This time of a (set of) form(s) and a meaning.\footnote{a ``formeme'' and a ``sememe'' by Filipec} For the forms the argument is stated above. So let us suppose that a foreign word (? or formeme(=set of forms)) is not a part of a lexeme. We have shown that there is no reason for this exclusion in the form(eme), so it must lie in the sememe, i.e. the meaning part of the pair. But what is the meaning (without a form!) and if it exists (without a form, let's keep that in mind), how can it limit the set of forms? We do not know, and neither do we know of anyone giving any reason for such limitation. In fact we do not know of anyone giving analysis of this problem. It seems that the only ``reason'' to exclude foreign forms (as long as they have the same meaning (however vague that is, but we will return to this momentarily in \Sref{sec:meaning-stability})) is an axiom that a lexeme is a component of a language $A$. A foreign form $F$ is a part of a language $B$, hence it is not a part of a language $A$, so it cannot be a part of any of its components.

\subsection{A meaning of a lexeme -- a sememe}
\label{sec:meaning-stability}
Let us now examine the stability of the meaning of a lexeme, because it seems to be the only possible reason to exclude both synonyms (Are they other formemes, or other lexemes?) and foreign forms (that can actually be seen as just more synonyms). To exclude them from a lexeme we must postulate that any and all synonyms (in any and all languages, including language of origin $A$ have a different meaning. That is however quite a courageous postulate.
{\xxx doplnit}

%%%%% MWE %%%%%%%
\section{MWE}

\citet{baldwin:2004} defines MWEs very broadly as entities that are:
\begin{itemize}
\item
``decomposable into multiple simplex words,'' and
\item
``lexically, syntactically, semantically, pragmatically and/or statistically idiosyncratic.''
\end{itemize}

His examples are as follows: \emph{``San Francisco, ad hoc, by and large, Where Eagles Dare, kick the bucket, part of speech, in step, the Oakland Raiders, trip the light fantastic, telephone box, call (someone) up, take a walk, do a number on (someone), take (unfair) advantage (of), pull strings, kindle excitement, fresh air, \ldots''}

From the definition and the examples it is clear that Baldwin includes not only idioms and complex verbs, but also any named entities and even any statistically or pragmatically important\footnote{We avoid a MWE (sic!) ``statistically significant'' on purpose, because we assume that Baldwin also avoids it on purpose when using a word ``idiosyncratic''. As far as we know ``statistically idiosyncratic'' is not a well defined term byt we understand it as saying that not any statistically significant difference in distribution is peculiar enough to be called ``idiosyncratic''. We are fully aware how imprecise this sounds.} collocations. At least that is what we understand as ``statistically idiosyncratic''. Such expressions include ``environmental policy'' but also ``salt and pepper'', which is semantically quite compositional and simple, but statistically the order of its components is significant. In the Corpus of Contemporary American English (COCA, \url{http://www.americancorpus.org/}), there are 3648 occurrences of ``salt and pepper'' vs. 62 occurrences of ``pepper and salt''. Of the 62 occurrences 60 are in recipes. This is rather extreme case of ``statistical idiosyncrasy''; as such it well illustrates the point.

Such a broad definition basically says that MWEs are ``interesting collocations'' but in its broadness it is not suitable for our purpose. Since we base our work on the concept of (monosemic) lexeme, we are more interested in the more conventional approach that Baldwin has in most of his other (co-authored) papers \citep{baldwin:2003,sag:2002}. MWEs are viewed as ``cohesive lexemes that cross word boundaries''. This seems to be the most common definition of MWEs in NLP, as long as we abstract from subtle differences in terminology \citep{calzolari, copestake, dalsi?}. {\xxx je to taky shodne s tradicnimi lingvisty? Cruse? Lyons? cesti?}


\section{ToAdd}
Pecina, Cruse, Filipec, Čermák (rozdelit, nebo spojit se ZŽ?), Hanks. Zminit clanek SC, MH a Lenky?? o Hanksove CPA jako realizaci Sinclairova pristupu k lexikalnimu vyznamu.

Valencni slovniky a \emph{anotace valence}.