% !TEX root = ../disertace.tex
%!TEX encoding = UTF-8 Unicode

\chapter{Words, lexemes, lexical units, meanings, and such}
Since we want to write about multiword expressions, named entities and annotation of instances of these in texts, we must define these concepts. In order to do so, however, we must examine more basic concepts: to define a multiword expression we should explain what we mean by expression. %
\footnote{Unless we say that ``multiword expression'' is an idiom and therefore the word ``expression'' is not a semantic unit therein. See Cruse's explanation in \Sref{rel:cruse}.
} %

Even worse, it seems  that we cannot escape at least touching the fear-inducing basic concept of pr{meaning}. And actually, we would not even want to. 

We will examine several works that influenced our approach and then we shall put forward our concepts as they will be used throughout the text.

\section{General}
\textbf{Framework?}\\
Do we work, i.e. also define all our concepts, in the framework of a/one language? This question is quintessential. Let us illustrate it on discussion of meaning (Bedeutung) in the semantic, syntactic, pragmatic trichotomy in \citet{sgall-etal:1986}: ``\ldots meaning itself incorporates incorporates pragmatic as well as semantic aspects. [\ldots] On the other hand, by far not all semantic differences are overtly included in the meaning patterns of a particular natural language. Suffice it to recall here that e.g. the distinction of G(erman) \emph{gehen} vs. \emph{fahren} is absent from the meaning of E. \emph{go}, just as the dual is lacking in the E. category of number.''

This quote clearly shows a concept of meaning before a language, or maybe above the set of all the languages. This may seem unimportant, but it can have a significant influence for instance on the definition of lexeme, as we will discuss shortly.  \xxx{dokoncit} 

\section{Cruse}
\label{rel:cruse}
\citetext{D. A. Cruse in his book \emph{Lexical Semantics}, \citeyear{cruse:1986}} has a thorough treatment of the basic concepts we are going to use throughout our text, so we shall start with this work. Also, some other authors whose work we will examine later refer to Cruse's work. 

\subsection{Aiding the intuition}
In the chapter on intuitive judgement Cruse puts forward an interesting observation that it is often the case in science to use human judgement indirectly. Since what we want to judge very often cannot be judged by humans reliably, the judgement is often transformed to a different judgement (and possibly another one, \ldots) that finally can be made by humans with sufficient reliability. It is important to see that when we say that we do X in order to eliminate human judgement, it is very often the case that we move the human judgement from the problem X to the problem Y. That is not to say it is a bad thing: it is the right thing to do as long as it increases reliability of the judgement. It is however important to keep in mind that it is not elimination of human judgement. {\xxx priklady: i statisticke testovani hypotez je mozna vhodny (extremni) priklad.}

\subsection{Context, intuition, normality, and meaning}
That some human intuition is needed has already been shown. Cruse however does not attempt to limit the intuition too strictly. The core intuition he argues for is native speakers' identification of \emph{normality}. 

\subsubsection{Lexical unit}
Cruse first establishes a notion of \emph{``a minimal semantic constituent''} and uses that notion in definition of an idiom in the following way.


\section{Sinclair}
Patrick Hanks gives a very succinct introduction to John Sinclair's \citep{sinclair:wiki} view of a lexical unit and meaning distinctions. It is closely related to Hanks' own work, as we can see in \citet{hanks:norms}.

\section{Žabokrtský}
Zdeněk Žabokrtský defines a \emph{lexeme} in his doctoral thesis \citep{zabokrtsky:2005a}, as well as several other basic concepts that he requires for precise description of his work on valency lexicon of Czech verbs. His definition is based on the concept of lexeme as defined in works of \citet{cermak:91} and \citet{filipec:1994} and a concept of lexical unit as defined by \citet{cruse:1986}\footnote{Žabokrský cites \citealp{verspoor:1997}'s description of Cruse's lexeme, but that makes no difference {\xxx(check!)}}. Žabokrtský's view seems to differ from the view of the above mentioned authors, but that difference deserves some analysis.

Žabokrtský defines a lexeme as ``an ordered pair which couples the set of lexical forms and the set of lexical units''. This way a definition of lexeme is based on a definition of the concepts ``lexical form'' and ``lexical unit''. It is also implied that these are somehow complementary. 

A lexeme is however a relatively common concept in lexicography. It has been defined at least by Lyons, by Cruse, Čermák, Filipec, and probably by many other authors. What is common to all of these definitions is that they construe lexeme as a unit consisting of some lexical form and meaning. 
\emph{It is in fact interesting to try to find what distinguishes a concept of lexeme from a concept of concept itself! It seems that a concept is also a meaning (essentia) with a form.} See \citet{materna:1998,stranak:2001} for detailed discussion of definition of concepts as well as thorough explanation (in the latter) of the reasons why we do not delve into the problem of concept itself too deeply: it is out of the methodological reach of modern science as we understand it: the problem of what is a concept cannot be solved by any conceptual analysis.

A \emph{lexical unit} in turn is used as defined by Verspoor referring to \citet{cruse:1986}.\footnote%
\xxx{Najit Crusovu definici lexemu a lex. unit a analyzovat. Ma tam MWE, nebo NE??} 

%% ME %%
\section{Straňák}
Concept vs. word/lexeme/lexia\\
ergo\\
Language vs. conceptual system\\
Is language in fact the conceptual system? If not, then why? What is the difference? 

\subsection{How about translations}
Are they a part of a lexeme? Why not?

To Žabokrtský, but also all(?) the lexeme definitions: A ZZ lexeme is an ordered (why??) pair composed of a set of forms and a set of lexical units (meanings in his understanding?). There is no limit on forms that can be a part of a lexeme, so there seems to be no reason to exclude the forms of a different language, as long as they fit the ``lexical units''. LU (or a lexeme of Filipec and Cruse(?)) is again a pair: This time of a (set of) form(s) and a meaning.\footnote{a ``formeme'' and a ``sememe'' by Filipec} For the forms the argument is stated above. So let us suppose that a foreign word (? or formeme(=set of forms)) is not a part of a lexeme. We have shown that there is no reason for this exclusion in the form(eme), so it must lie in the sememe, i.e. the meaning part of the pair. But what is the meaning (without a form!) and if it exists (without a form, let's keep that in mind), how can it limit the set of forms? We do not know, and neither do we know of anyone giving any reason for such limitation. In fact we do not know of anyone giving analysis of this problem. It seems that the only ``reason'' to exclude foreign forms (as long as they have the same meaning (however vague that is, but we will return to this momentarily in \Sref{sec:meaning-stability})) is an axiom that a lexeme is a component of a language $A$. A foreign form $F$ is a part of a language $B$, hence it is not a part of a language $A$, so it cannot be a part of any of its components.

\subsection{A meaning of a lexeme -- a sememe}
\label{sec:meaning-stability}
Let us now examine the stability of the meaning of a lexeme, because it seems to be the only possible reason to exclude both synonyms (Are they other formemes, or other lexemes?) and foreign forms (that can actually be seen as just more synonyms). To exclude them from a lexeme we must postulate that any and all synonyms (in any and all languages, including language of origin $A$ have a different meaning. That is however quite a courageous postulate.
\xxx{doplnit}
