% !TEX root = ../disertace.tex
\chapter{\sdata}
\section{The design and the PML schema}
\sdata\ means s-layer PML files and the PML schema of these files. The idea behind \sdata\ design is to have a simple way to store additional ``sense'' annotations over any layer of PDT. The annotations are stored as a set of ``sense'' nodes where each s-node contains a link to a sense repository (annotation dictionary) and a set of references to nodes (m-, a- or t-) that correspond to an instance of the sense. An \sf\ is thus basically a very simple flat list of \sn{}s. It does not contain any trees. A single \sf\ can only reference a single PDT file: either tectogrammatical, or analytical, or even morphological layer can be used, but references to different layer cannot be mixed.

\section{Visualisation}
There are two basic ways to view st-nodes: in \seman\ or in \tred. Both of these need to use the ``t-a-m-w-'' PDT files to display the sentence and/or the tree for each sentence and then they read the \stf\ to add the information about \stn{}s. The \stn{}s are displayed as colour boxes or bubbles over the words in a sentence or nodes in a tree in \seman\ or \tred\ respectively.

\section{\seman}
The visualisation of annotated files in \seman\ has the advantage of showing whole text with all the \mwe{}s clearly marked in a single glance. Integration of the SemLex browser is also beneficial, because it allows fast and convenient lookup of annotated \mwe{}s in \seman. Details of \seman\ interface are described in \Sref{sec:seman}. 

There are, however, also some drawbacks of this ``full plain text of an article'' approach: 

